\acknowledgments {I'd like to thank my advisor, Michael Alfaro, for taking me under his wing and teaching me the ropes of big data research. His support, feedback, and mentorship taught me how to be a critical scientist and ask questions that move the field forward. Prior to entering his lab --- first as an undergraduate research assistant in Fall 2018 prior to beginning my master's in Fall 2019 --- I lacked confidence in my ability to succeed as a scientist in academia. It was during Mike's \emph{Biology and Social Justice} (EE BIOL 156) seminar in the Summer of 2018 where we first met and I was captivated by the perspectives he put forth during our discussions. Mike made me think harder and deeper about the intersection of critical social issues with core biological concepts than I had ever done before. Had we not crossed paths then, I would have not had the opportunity to serve as the teaching assistant, and then associate, for EE BIOL 156 three times over during my master's --- one of my most fond memories of graduate school at UCLA. Mike, thank you for investing in me as an evolutionary biologist.

To my committee members, Greg Grether and Felipe Zapata, and additionally to my proposal reader, Nathan Kraft --- thank you all for your inspiration and guidance, especially in helping shape this research during its nascent stages.

To Alma, my lovely wife and best friend, for your tireless support, motivation, and inspiration --- especially when times were tough. I don't think I'd be where I am today had we not sat next to each other on the first day of our first-year neuroscience Fiat Lux seminar at UCLA in Spring 2016. Since then, you've taught me how to push the limits of what I thought was possible and to never give up when in the moments I was the weakest. From friendship and love to marriage, you've never once given up on me nor have you let me give up in the times I most wanted to --- I am forever grateful for this. During what seemed like a fun stay-at-home vacation at the beginning of the global COVID-19 pandemic in March of 2020 quickly turned out to be much scarier and less relaxing as we initially thought, but together we learned more about each other and began to cherish the fragility of life. I can't image spending those many months of isolation with anyone other than you and our Corgi puppy, Reese, who together form our beautiful and joyful family. Finally, you and Reese have taught me how to cope with my anxieties and other mental health issues --- especially during this pandemic --- and this immensely helped me keep my head up in the most challenging of times while completing the research presented in this thesis. I love you and can't thank you enough for helping shape me into a person I'm proud to be today.

I'd next like to thank my parents for their guidance and support during my childhood into adulthood. Specifically, thank you to my mom, Terri, for never hesitating to stay up late with me while I completed my homework or taking me around town when I needed a lift to school or the science fair. To my dad, Bruce, for making time to spend with us despite working on-call, 24/7 in a blue-collar profession. I would not be the person I am today without their love and care. I'd also like to thank my sister, Nicole, for encouraging me to get up from the computer and play outside every once and a while when we were kids. Her artistic creativity has been a source of inspiration for me in the times I've lacked creativity myself. I'd also like to thank my brother and sister in-laws, Rafael and Laura, for always being available to give advice and make me laugh with a joke or funny video when I've needed it most. Rafael, as we've embarked on our graduate school journey together, I'm so excited and enthusiastic to see where our careers take us, and I couldn't have asked for a better brother, and friend, to grow professionally with as first-generation students in academia. Por último, a mis suegros, Teresa y Rafael, por inspirarme a aprender español, apreciar la vida y disfrutar de las hermosas y ricas comidas de Chavinda, Michoacán, México.

Next, to those who were instrumental in my pedagogical growth as an instructor, especially the UCLA Center for the Integration of Research, Teaching, and Learning (CIRTL@UCLA) and the UCLA Center for Education Innovation \& Learning in the Sciences (CEILS). I'd specifically like to thank Rachel Kennison, Leigh Harris, Katie Dixie, and Elizabeth Reid-Wainscoat for their support and mentorship, in addition to the UCLA CIRTL Teaching-as-Research (TAR) community, especially Manisha Chase, Shawn McEachin, Annie Wofford, Amelia Hill, Letty Treviño, and Chelsea Romney for their support and engagement during the sudden shift to remote instruction at the onset of the COVID-19 pandemic --- thank you \emph{CIRTL turtles} for always being such a welcoming and positive community! To my other instructional mentors --- Michael Alfaro, Pamela Yeh, Rachel Prunier, Iris Firstenberg, Amber Ankowski, Elizabeth Bjork, Alan Castel, Tyler McCraney, Ginny Sklar, Manisha Chase, and Mary Whatley  --- for showing me the importance of compassion and flexibility through your passion for teaching and care for students; you've all played a role in shaping me as the teacher I am today.

Lastly, I'd like to thank all of those who --- directly and indirectly --- contributed to the research presented in this thesis. To Elizabeth Karan, Whitney Nakashima, and Mark Juhn, for many insightful and inspirational conversations about color pattern diversity, mechanisms, and deep learning to helping me find my way when hitting programming walls. I'm grateful for our friendship born out of the many in-person discussions prior to the pandemic, in addition to the fun virtual social gatherings on Zoom that made doing science while sheltering in place less isolating. To Tyler McCraney, for always making me feel welcome and for useful advice about how to ask interesting and fun scientific questions that excite and motivate me. To Christiane Jacquemetton and Jonathan Chang, for helpful grad school, professional, and life advice that I'll take with me to each stage of my professional career. To Alex Siegel, for being an awesome friend and collaborator, and for being instrumental in teaching me more advanced statistical modeling methods and leading me to find my true research passions studying the neuroscience of memory and cognitive aging. To Alan Castel, for your mentorship and guidance in research, especially in helping me conduct my first independent research study during undergrad and guiding me to where I am today as a professional in academia. To my many collaborators and coauthors --- both past and present --- especially, Allison Shultz, Lars Schmitz, Katie Silaj, Mary Whatley, Dillon Murphy, Hannah Weller, Mary Hargis, Lauren Richmond, Julia Kearley, Ian McDonough, and Neil Garg. To the wonderful research assistants who've been instrumental in getting some of this (and other) research off the ground and for reminding me of the joy and inspiration that comes with mentorship: Mackenzie Perillo, Maya Chari, Aubrey Butler, Kevin Wang, Jonathan Chau, Nimrat Brar, Trevor Brokowski, Kate Javerbaum, Austyn Adams, Kyle Xu, and Charles Wang.

The work presented in this thesis is a reprint of: Shawn T. Schwartz, \& Michael E. Alfaro (2021). \emph{Sashimi}: A toolkit for facilitating high-throughput organismal image segmentation using deep learning. \textit{Methods in Ecology and Evolution, 12}, 2341–2354. \href{https://doi.org/10.1111/2041-210X.13712}{doi:10.1111/2041-210X.13712}. S.T.S. and M.E.A. conceived of the study; S.T.S. wrote the software. Both authors contributed to the writing of the manuscript.}