\abstract       {Utilizing the comparative method at massive analytic scales requires the acquisition of large samples of characters for taxa across the tree of life. When data acquisition approaches are limited, studies may subsequently limit their analysis to a particularly conserved group of taxa to avoid issues of incomplete sampling at larger phylogenetic scales. The inherent difficulty associated with obtaining large datasets imposes a data bottleneck for studying comparative macroevolution through deep time scales. Having access to powerful and flexible artificially intelligent approaches for data acquisition and pre-processing are therefore important for facilitating larger scales of analysis. Machine learning provides unprecedented opportunities to exploit massive datasets. The subsequent development of deep learning applications specialized for automating cumbersome human tasks is possible given that these models learn over time to perform such tasks with accuracy similar to that of a human observer. Deep learning is a branch of machine learning that holds enormous potential for ecologists and evolutionary biologists in an era of research becoming increasingly reliant on big data. These tools can streamline data extraction from field observations and recordings, in addition to uncovering complex patterns in dense multivariate datasets. Here, I focus on leveraging deep learning as a toolkit for image segmentation. This toolkit,  \emph{Sashimi}, provides a reproducible, rapid, and automated approach for pre-processing digitized images of organisms necessary for downstream analyses of visual phenotypes --- such as color patterns --- at massive phylogenetic scales.}